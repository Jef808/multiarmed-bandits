% Created 2022-09-14 Wed 11:01
% Intended LaTeX compiler: pdflatex
\documentclass[11pt]{article}
\usepackage[utf8]{inputenc}
\usepackage[T1]{fontenc}
\usepackage{graphicx}
\usepackage{longtable}
\usepackage{wrapfig}
\usepackage{rotating}
\usepackage[normalem]{ulem}
\usepackage{amsmath}
\usepackage{amssymb}
\usepackage{capt-of}
\usepackage{hyperref}
\author{Jean-Francois Arbour}
\date{\today}
\title{Noise Generation}
\hypersetup{
 pdfauthor={Jean-Francois Arbour},
 pdftitle={Noise Generation},
 pdfkeywords={},
 pdfsubject={},
 pdfcreator={Emacs 28.1 (Org mode 9.6)}, 
 pdflang={English}}
\usepackage{biblatex}
\addbibresource{~/bib/references.bib}
\begin{document}

\maketitle
\tableofcontents


\section{Gaussian Noise generator}
\label{sec:org3af9a6b}
Two simple ways of generating samples from a Gaussian distribution:
\subsubsection{Box and Muller method}
\label{sec:org1429bae}

\begin{theroem}\label{thrm:box-muller}\cite{Box1958ANO}
Let $u$ and $v$ be independant samples chosen from a uniform distribution on $(0, 1) \subset \mathbb{R}$ and $(R, \Theta)$ corresponding polar coordinates.
Let
\begin{equation}
  z_0 = R \cos (\Theta) = \sqrt{-2 \log u } \cos (2 \pi v)
\end{equation}
and
\begin{equation}
  z_1 = R \sin (\Theta) = \sqrt{-2 \log u } \sin (2 \pi v).
\end{equation}
Then $z_0$ and $z_1$ are independant random variables both distributed a standard normal distribution of mean $0$ and variance $1$.
Generate samples from a standard normal distribution with mean $\mu$ and variance $\sigma^2$ by considering the random variables $Z_{0,1} = \mu + \sigma * z_{0,1}$.
\end{theorem}

\subsubsection{Marsaglia method}
\label{sec:org024a80b}
Let us first establish the geometric picture for this result:
Consider the open square
\mathbb{Sq} = (-1, 1) \times (-1, 1)

of side length \(2\) in the plane, along with its inscribed open disk
\mathbb{D} = \{ (x, y) \mid x ^ 2 + y ^ 2 < 1 \}.

For \(u\) and \(v\) two independant random variables with values in the plane, we will denote in what follows by
\begin{equation*}
\rho = \rho(u, v) = u ^ 2 + v ^ 2
\end{equation*}

Consider any two uniformly distributed and independant variables \(U, V\) with values in \(\mathbb{Sq}\) but constrained by the condition that $0 < \rho(U, V) < 1$ with probability $1$.
Then with
\begin{subequations}\begin{equation}\label{eq:z0}
  z_0 = \frac{U}{\sqrt{ \rho }} * \sqrt{ \frac{ -2 \,\log{\rho} }{\rho} }
\end{equation}
and
\begin{equation}\label{eq:z1}
  z_1 = \frac{V}{\sqrt{ \rho }} * \sqrt{ \frac { -2 \,\log{\rho} }{\rho} }
\end{equation}\end{subequations}
\begin{remark}
Notice that
\begin{equation*}
  \frac{U}{\sqrt{\rho}} = \cos(\theta)
\end{equation*}
and similarly,
\begin{equation*}
  \frac{V}{\sqrt{\rho}} = \sin(\theta).
\end{equation*}
So by writing $R = \displaymath{\sqrt{\frac{-2 \log { \rho }}{\rho}}}$, we can write
\begin{equation}\begin{cases}
z_0 &= R \cos (\theta), \\
z_1 &= R \sin (\theta).
\end{cases}\end{equation}
\end{remark}

\begin{theroem}[\cite{MR172441}]
With the notation as above, $z_0$ and $z_1$ form two independant variables distributed along a standard normal distribution with mean $0$ and standard deviation $1$. By scaling them to
\begin{equation}\begin{cases}
Z_0 &= \mu + \sigma z_0, \\
Z_1 &= \mu + \sigma z_1,
\end{cases}\end{equation}
we obtain two independant variables distributed along a normal distribution with mean $\mu$ and variance $\sigma ^ 2$.
\end{theorem}

\subsection{Test}
\label{sec:org0e4ef8e}
\subsubsection{Box and Muller method}
\label{sec:org342b0cc}

\begin{verbatim}
import matplotlib
matplotlib.use('Agg')
import matplotlib.pyplot as plt
import numpy as np
\end{verbatim}

As a test, we'll generate samples from a normal distribution using numpy's blackbox:
\begin{verbatim}
mu = 0.0
sigma = 1.0
number_of_samples = 200
normal_samples = np.random.default_rng().normal(mu, sigma, number_of_samples)
\end{verbatim}

Sanity checks:
\begin{verbatim}
mean_near_mu = abs( mu - np.mean(normal_samples) )
std_near_sigma = abs(sigma - np.std(normal_samples, ddof=1))

assert(mean_near_mu < 0.2)
assert(std_near_sigma < 0.2)
\end{verbatim}

Plot the above
\begin{verbatim}
fig=plt.figure(figsize=(4,2))
count, bins, ignored = plt.hist(normal_samples, 30, ec='orange', fc='steelblue', density=True)
plt.plot(bins, 1/(sigma * np.sqrt(2 * np.pi)) *
               np.exp( - (bins - mu)**2 / (2 * sigma**2) ),
         linewidth=2, color='violet', alpha=0.5)
fig.tight_layout()
fname="images/numpyExpectationsBarChart.png"
plt.show(fname)
#fname
\end{verbatim}

Compare with data generated from our own gaussian sampling method:
\begin{verbatim}
import math
import random

def getGaussianPair():
    u = random.random()
    v = random.random()
    while u == 0:
        u = random.random()
    sqrt_minustwo_logu = math.sqrt(-2 * math.log(u))
    two_pi_v = 2 * math.pi * v
    z_0 = sqrt_minustwo_logu * math.cos(two_pi_v)
    z_1 = sqrt_minustwo_logu * math.sin(two_pi_v)
    return [z_0, z_1]

def sample():
    if (self.has_spare):
      self.sample.append(self.spare)
      self.has_spare = False
    else:
      self.spare, result = getGaussianPair()
      self.has_spare = True
      self.sample.append(result)

def getSample(sample_size):
    result = []
    for _ in range(sample_size):
        result.append(getGaussianPair()[0])
        print(result)
\end{verbatim}

\begin{verbatim}
sample = getSample(number_of_samples)

fig=plt.figure(figsize=(4,2))
count, bins, ignored = plt.hist(sample, 30, ec='orange', fc='steelblue', density=True, histtype='barstacked')
plt.plot(bins, 1/(sigma * np.sqrt(2 * np.pi)) *
               np.exp( - (bins - mu)**2 / (2 * sigma**2) ),
         linewidth=2, color='violet', alpha=0.5)
fig.tight_layout()
fname="images/expectationsBarChart.png"
plt.show(fname)
fname
\end{verbatim}
\end{document}
